\documentclass{pre-tfg}
\usepackage{url}
%\showhelp  % comenta o borra para eliminar ayudas

\title{Sense: Plataforma online de almacenamiento, visualización y analisis
  de datos para `Internet of Things`.}
\author{David Martín García}
\advisorFirst{David Villa Alises}
\advisorDepartment{Tecnologias y sistemas de información}
\advisorSecond{Maria Jose Santofimia Romero}
\intensification{COMPUTACIÓN}
\docdate{2015}{Noviembre}


\begin{document}

\maketitle
\tableofcontents

\newpage

\section{INTRODUCCIÓN}

La sociedad avanza hacia la una nueva era en la que internet forma un
gran papel. En los últimos años estamos experimentando un crecimiento exponencial en la cantidad de datos volcados en internet así como en el numero de dispositivos que crean o consumen tal información. Actualmente cursamos dentro de la intensificación de computación asignaturas como 'Sistemas Multiagentes' o 'Minería de Datos' donde principios como la ubicuidad o el big-data quedan marcados como una premisa de futuro.

Todo esta interconectado y tecnologías como la domotica, ocio o la
salud sufren esta tendencia. Hoy en día dispositivos como smart
watches, smart bands, smart phones, smart tv se conectan a internet
para consumir y publicar datos, desde tendencias de uso hasta datos
personales de nuestra salud o ejercicio físico. Esta ingente cantidad
de datos conlleva grandes problemas de almacenamiento, así como un
requisito, la optimalidad, ya que el numero de conexiones crece
exponencialmente, tanto que es necesario nuevos protocolos como IPv6,
donde existen 340 sextillones de direcciones. Esto hace intuir que no
es una simple moda o tendencia que se disipara en unos años, es una
necesidad, y empresas como
IBM\footnote{http://www.ibmbigdatahub.com/technology/internet-of-things},
Microsoft o
Intel\footnote{https://software.intel.com/es-es/iot/microsoft-azure}
apuestan por ello.

Por otro lado, la mayoria de hubs existentes siguen un modelo
freemium
\footnote{https://azure.microsoft.com/en-us/pricing/details/iot-hub/}
\footnote{https://devicehub.net/about/pricing}
\footnote{http://community.thingspeak.com/documentation/api/}
\footnote{http://xively.com/whats\_xively/}
orientado a empresas y cuyas condiciones, si existen, son
insuficientes para la gran mayoría de proyectos o incluso limitadas a
determinadas placas\footnote{https://www.artik.io/}. También debemos
valorar la integracion con la mayor candidad de dispositivos
disponibles, asi como la gran ventaja del software libre, donde tu proyecto puede ser llevado en conjunto por una gran
comunidad de desarrolladores que mejoran y añaden funciones a la
aplicación final.

En este punto comienza este proyecto, donde se pondra en marcha un Hub de datos, donde poder publicar y consumir toda esta información. La
principal meta de este proyecto es la optimizacion de este proceso, asi como el analisis de los datos, dos de los principales hitos en la
mayoria de Hubs actuales. Este hub sera de caracter gratuito y open-source, donde se mezclan tecnologias y lenguajes de forma
heterogenea, y donde cada uno es el mejor en su campo, todos ellos en conjunto formaran un conjunto de micro-servicios, interrelacionados
entre ellos.

El uso de bases de datos NoSQL es una de las principales bazas dentro
de este proyecto, estas bases de datos nos permiten ingerir y realizar
operaciones a un bajo costo, asi como permitirnos realizar analisis de
los datos desde la propia base de datos. Este tipo de bases de datos son usada por entidades como el CERN, así como un innumerables empresas en internet (Ebay, GitHub, Instagram, reddit, GoDaddy, Twitter, Facebook...). Su escalabilidad y su
rendimiento es la principal razon por la que se han seleccionado, en
concreto hacemos uso de Cassandra, la base de datos que empresas como
Twitter\footnote{http://www.datastax.com/2014/04/good-morning-manhattan}
y Facebook han hecho uso, o incluso han modificado para sus propias
necesidades.

Otra de las necesidades dentro de este proyecto sera realizada en el
sucesor de uno de los lenguajes mas eficientes dentro de las
tecnologias web, donde una empresa lider es uno de los mejores
ejemplos de lo que se puede realizar mediante este lenguaje, hablamos
de Erlang, cuya eficiencia en entornos multi-hilo no tiene precedente,
uno de los mejores casos de uso se encuentra en WhatsApp, empresa que
mantuvo sus servicios con un unico servidor y 50 ingenieros todo su
potencial\footnote{http://www.fastcompany.com/3026758/inside-erlang-the-rare-programming-language-behind-whatsapps-success}\footnote{https://blog.whatsapp.com/196/1-millón-es-tan-2011?}. Su origen esta marcado por las telecomunicaciones y en
principal por Ericcson, donde alcanzo una tasa de fiabilidad y uptime
del 99,999999999\%, esta cifra nos hace intuir que es un lenguaje robusto, fiable, y muy eficiente en su campo, donde la eficiencia es su maximo exponente.

Otras tecnologias al uso sera Docker, que nos permitira implantar un
entorno de desarrollo estable y controlado asi como la facilidad para migrar a un modelo cloud.
Se hara uso de containers como unidad, permitiendonos tener
aplicaciones aisladas que trabajan conjuntamente. Este entorno de
podra regenerar automaticamente mediante el aprovisionamiento de sus caracteristicas.

\section{TECNOLOGÍA ESPECÍFICA / INTENSIFICACIÓN / ITINERARIO CURSADO POR EL ALUMNO}

\begin{table}[hp]
  \centering
  \caption{Tecnología Específica cursada por el alumno}
  \label{tab:tec-especifica}

  \zebrarows{1}
  \begin{tabular}{p{0.01\linewidth}p{0.4\linewidth}}
    &\textbf{Marcar la tecnología cursada} \\
    \hline
    & Tecnologías de la Información \\
    \textbf{*} &\textbf{Computación} \\
    &Ingeniería del Software \\
    &Ingeniería de Computadores \\
    \hline
  \end{tabular}
\end{table}


\begin{table}[hp]
  \centering
  \caption{Justificación de las competencias específicas abordadas en el TFG}
  \label{tab:competencias}

  \zebrarows{1}
  \begin{tabular}{p{0.5\linewidth}p{0.5\linewidth}}
    \textbf{Competencia} & \textbf{Justificación} \\
    \hline
    Capacidad para adquirir, obtener, formalizar y representar el conocimiento humano en una forma computable para la resolución de problemas mediante un sistema informático en cualquier ámbito de aplicación, particularmente los relacionados con aspectos de computación, percepción y actuación en ambientes entornos inteligentes.
    & Al tratar con valores de cualquier forma, como por ejemplo
    sensores, debemos comprender las diferentes magnitudes asi como una
    posible forma normal para poder comparar entre los distintos valores.\\
    Capacidad para desarrollar y evaluar sistemas interactivos y de
    presentación de información compleja y su aplicación a la resolución
    de problemas de diseño de interacción persona computadora.
    & Desarrollo de entorno grafico donde poder monitorizar los datos
      asi como las distintas opciones y incidencias que se detecten,
      ademas de la gestion de perfiles de usuario asi como de los
      distintos metadatos asociados a la entrada de datos.\\
    Capacidad para conocer y desarrollar técnicas de aprendizaje computacional y diseñar e implementar aplicaciones y sistemas que las utilicen, incluyendo las dedicadas a extracción automática de información y conocimiento a partir de grandes volúmenes de datos.
    & Analisis de los datos obtenidos asi como  deteccion de valores anormales y obtencion de posibles respuestas ante tal estimulo.\\
    Capacidad para evaluar la complejidad computacional de un problema, conocer estrategias algorítmicas que puedan conducir a su resolución y recomendar, desarrollar e implementar aquella que garantice el mejor rendimiento de acuerdo con los requisitos establecidos.
    & Diseño y implementación de algoritmo de agrupamiento de datos. Este algoritmo agregara los diferentes datos con el fin de mantener controlado el tamaño máximo de la base de datos.\\
    \hline
  \end{tabular}
\end{table}
 \clearpage

\section{OBJETIVOS}

Desarrollo de un hub para el almacenamiento de datos con caracter
temporal orientado al concepto IoT (`Internet of Thing`).

Se debe plantear un arquitectura de microservicios de alto
rendimiento. Se plantea una parte central que permita el
almacenamiento, analisis y tratamiento de los datos, asi como
servicios en torno a esta para las notificaciones a los distintos dispositivos que requieran de estos datos.

Debe existir servicios de notificación, analisis, autentificación
ademas de la gestion y almacenamiento de los datos asi como su consulta.

Tambien se require de una interfaz grafica de gestion que se desarrolla
mediante una aplicación web, asegurando asi su total compatibilidad
con la gran mayoria de dispositivos existente.

Debe tener compatibilidad con los estandares de comunicacion REST e
ZeroC ICE, teniendo ademas en cuenta MQTT como posible candidato.

Se debe disponer de una API para la comunicacion con los distintos
dispositivos externos, asi como un protocolo de comunicación para la
parte interna entre servicios.

Se tendra el cuenta el rendimiento como principal punto de seleccion
en la infraestructura asi como en cada parte del proyecto, debido a
esto se preve un desarrollo multilenguaje como punto principal para la
busqueda de este objetivo.

Tambien se desarrollara el aprovisionamiento de servidores asi como la
disposicion de un entorno de desarrollo de facil instalación para
cualquier desarrollador.

TODO Incluir grafico preliminar de la arquitectura

\section{MÉTODO Y FASES DE TRABAJO}
Para llevar a cabo este proyecto se ha llevado a cabo un análisis de
las diversas metodologías existentes.

La eleccion de una metodologia agil es fiel reflejo de la naturaleza
de este TFG, donde a fecha inicial no se define totalmente los
requisitos con total exactitud y donde la planificacion a corto plazo
cobra un mayor sentido.

Tras obtener las ventajas e inconvenientes de cada una se ha optado por scrumban, una mezcla
entre Kanban y Scrum, donde se pactaran diferentes hitos dentro del
plazo máximo para desarrollar el proyecto, cumpliendo con diferentes
sprints, pero a su vez haciendo uso de Kanban como planificación de
las diferentes tareas.

Con esto tenemos las ventajas de ambos, por un lado controlamos el
progreso en casa sprint, y nos permite a su vez gestionar las tareas
mediante un tablón Kanban, obteniendo la mínima latencia desde el
desarrollo hasta la integración con su versión de producción. Además
nos permite seguir las diferentes fases del software, desde el diseño,
hasta su puesta al publico, pasando por las fases de codificación,
test, aceptación y calidad.

Esta metodología nos permite adaptarnos a cualquier cambio en los
requisitos o en el ritmo de trabajo, ya que pro la propia naturaleza
del TFG, debe compaginarse con el resto de actividades academicas.

TODO: DEFINIR HITOS A GRANDES RASGOS, es por hacerme una idea del tamaño Corregir si o si.....

\begin{itemize}
\item 1º hito
\item - Definición de la arquitectura del proyecto
\item - Análisis de las diferentes partes
\item - Realización de la documentación inicial y puesta a punto de los
\item diferentes servicios involucrados en el desarrollo
\item 2º hito
\item - Analisis de rendimiento de las diferentes propuestas, asi como la
\item obtencion de una primera cifra del rendimiento esperado del proyecto en cada
\item una de sus partes.
\item - Codificación del núcleo de micro-servicios
\item - Obtención de la primera versión mínimamente funcional
\item 3º hito
\item - Desarrollo de la interfaz web
\item - Desarrollo del backend de análisis de datos.
\item 4º hito
\item - Integración con dispositivos.
\end{itemize}
\clearpage
\section{MEDIOS QUE SE PRETENDEN UTILIZAR}

\subsection{Medios Hardware}
Para la realización de este proyecto se van a utilizar los siguientes
medios hardware, se estructuraran en los diversos entornos
disponibles:


\textbf{Desarrollo}:
Entorno de desarrollo, donde se realizan las tareas de codificación.

\begin{table}[hp]
  \caption{Características del entorno de desarrollo.}
  \centering
  \zebrarows{1}

  \begin{tabular}{p{0.2\linewidth}p{0.4\linewidth}}
    & MacBook Pro 2014 Mid.\\
    CPU& Intel i5 4308U @ 2.8Ghz.\\
    RAM& 8GB DDR3L 1600Mhz.\\
    Almacenamiento& SSD M2 128GB.\\
    SO& Debian 8.2.\\
  \end{tabular}
\end{table}

\textbf{Staging}
Servidor local donde se realizaran las distintas pruebas a la
plataforma para comprobar su funcionamiento, ademas realiza las
funciones de servidor publico donde se alojara la plataforma.

\begin{table}[hp]
  \caption{Características del entorno de staging.}
  \centering
  \zebrarows{1}

  \begin{tabular}{p{0.2\linewidth}p{0.4\linewidth}}
    CPU& Intel i5 4690K @ 5Ghz.\\
    RAM& 16GB DDR3 1833Mhz.\\
    Almacenamiento& SSD 120GB + RAID0 2TB (2 discos de 1TB).\\
    SO& Debian 8.2.\\
  \end{tabular}
\end{table}

\textbf{Producción}
Servidor dedicado donde se encuentra el entorno publico, este entorno
es el que disfrutaran los usuarios.

\begin{table}[hp]
  \caption{Características del entorno de staging.}
  \centering
  \zebrarows{1}

  \begin{tabular}{p{0.2\linewidth}p{0.4\linewidth}}
    CPU& Intel Xeon E5 2620V2.\\
    RAM& 32GB DDR3 ECC.\\
    Almacenamiento& 3x SSD 256GB (RAID0) + 100GB (Backups).\\
    SO& Debian 8.3.\\
  \end{tabular}
\end{table}

\clearpage
\subsection{Medios Software}
\begin{itemize}
\item Control de versiones: Git.
\item Comunicación entre las distintas partes: Slack.
\item Técnicas de diseño e implementación software: TDD.
\item Servicios de integración continua: TravisCI ó CircleCI.
\item Lenguaje: Elixir, C, Python, Ruby, Javascript, Html5.
\item Entorno de desarrollo: Emacs.
\end{itemize}

Además, se prevé utilizar las siguientes tecnologías en la capa de
persistencia:

\begin{itemize}
\item Base de datos: NoSQL con escalabilidad en horizontal asi como un
  tiempo medio de respuesta bajo en las inserciones. Se prevé utilizar Cassandra.
\item Persistencia temporal: Redis.
\end{itemize}


\section{REFERENCIAS}

En esta sección se incluirán todas las referencias bibliográficas, ordenadas
alfabéticamente por el primer apellido del primer autor, de las obras de las cuales se
haya realizado alguna cita en los apartados anteriores. Las referencias deberán contener
datos básicos como nombre y apellidos de los autores, título de la obra, evento al que
pertenece, páginas, fecha y lugar de celebración (si se tratara de artículos de congreso),
ISBN, editorial y ciudad (si se tratara de libro), nombre de revista, páginas, volumen y
número (si se tratara de revista), etc.

Se empleará un formato de referencia reconocido en el ámbito académico como
ACM\footnote{http://www.acm.org/sigs/publications/proceedings-templates}\footnote{http://www.cs.ucy.ac.cy/\~{}chryssis/specs/ACM-refguide.pdf}.
Otros formatos aconsejables son, por ejemplo, IEEE, AMA, APA y AMA.

A continuación una sección de «Referencias» con ejemplos de referencias con formato ACM para:

\begin{itemize}
 \item Un artículo de revista~\cite{Bow93}.
\item Un informe técnico~\cite{Ding97}.
\item Un libro~\cite{Tavel07}.
\item Un capítulo de libro~\cite{Greiner99}.
\item Un artículo en las actas de un congreso~\cite{Frohlic00}.
\item Para una página web~\cite{Steele04} (con autores conocidos).
\item Para una página web~\cite{Oxygen} (con autores desconocidos).
\end{itemize}

\bibliographystyle{alpha}
\bibliography{main}

\end{document}

% Local Variables:
% coding: utf-8
% mode: flyspell
% ispell-local-dictionary: "castellano8"
% mode: latex
% End:


%http://www.erol.si/2015/01/the-complete-list-of-all-timeseries-databases-for-your-iot-project/
%http://www.infoworld.com/article/2825890/application-development/why-redis-beats-memcached-for-caching.html
%http://phoenix.thefirehoseproject.com/0.html
%https://github.com/trenpixster/addict
%https://github.com/opendrops/passport
%http://safecast.org/tilemap/
%https://es.wikipedia.org/wiki/IPv6
%http://cassandra.apache.org/
%http://stackoverflow.com/questions/410616/increasing-the-maximum-number-of-tcp-ip-connections-in-linux

%(IOT CASSANDRA - https://www.instaclustr.com/customers/iot/,
%https://spark-summit.org/2014/wp-content/uploads/2014/07/Using-Spark-Streaming-for-High-Velocity-Analytics-on-Cassandra-Albert-Tobey-Tupshin-Harper.pdf,
%http://www.datastax.com/internet-of-things,
%http://www.planetcassandra.org/blog/functional_use_cases/internet-of-things-sensor-data/,
%http://highscalability.com/blog/2010/7/11/so-why-is-twitter-really-not-using-cassandra-to-store-tweets.html,
%)

%  LocalWords:  Intel ademas
