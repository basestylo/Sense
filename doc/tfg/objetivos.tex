\chapter{Objetivos}
\label{chap:objetivos}

%\noindent
%Para este capítulo, la normativa indica:

%«Concretar y exponer el problema a resolver describiendo el entorno de trabajo,
%la situación y detalladamente qué se pretende obtener. Limitaciones y
%condicionantes a considerar para la resolución del problema (lenguaje de
%construcción, equipo físico, equipo lógico de base y de apoyo, etc.). Si se
%considera necesario, esta sección puede titularse ``Objetivos e hipótesis de
%trabajo''. En este caso, se añadirán las hipótesis de trabajo que el alumno, con
%su TFG, pretende demostrar».

\section{Objetivo general}

Este proyecto tiene el objetivo de poner en marcha una aplicación web, para esto
debemos llevar a cabo los principales procesos involucrados en la
creación de cualquier servicio web, desde el análisis inicial hasta su puesta en
marcha y conexión con el mundo, teniendo en cuenta el mantenimiento posterior a
tal momento.

Se busca desarrollar una plataforma de gestión de sensorización dentro del ámbito de IoT con un rendimiento
por encima de todo lo actualmente visto, teniendo tasas de tiempos de respuesta
cercanos a la milésima de segundo en los puntos vitales, además de aguantar un
gran número de conexiones concurrentes, esto hace que se aproveche al máximo los recursos de
la máquina huésped, además crearé un proyecto open-source y un entorno favorable para
que se genere una comunidad entorno a este.

Esta plataforma puede posicionarse como alternativa a otros servicios gratuitos,
gracias al bajo coste derivado de su alto rendimiento. No existe algo similar en
términos de simplicidad y rendimiento en este momento.

%El hito final que se pretende lograr, destacando el problema específico que
%resuelve o la funcionalidad que aporta la aplicación o sistema desarrollado.


\section{Objetivos específicos}

Podemos destacar varios apartados, que serán piezas básicas para la realización
de tal proyecto:

\subsection{Análisis y selección del conjunto de elementos del
proyecto:} Comparativa del rendimiento de los actuales frameworks web así como de la
infraestructura necesaria. Dado que una de las máximas de
este proyecto será un elevado rendimiento a un coste bajo, se seleccionará entre
todos los candidatos el que mejor se adapte ante esta situación para cada una de
las partes involucradas.

\subsection{Identificación y análisis de los datos ligados al internet de las
  cosas (IoT):} Identificar posibles estándares ligados a este concepto, además de desarrollar la
arquitectura básica que tomarán los datos en nuestra aplicación. Se requerirá
analizar plataformas existentes, así como indagar sobre posibles estándares a la
hora de enviar información.

\subsection{Crear un entorno para desarrolladores:}
Este entorno debe facilitar a cualquier desarrollador el poder aterrizar en el
proyecto con la mínima dificultad posible, debe ser totalmente estable y
totalmente aislado de la máquina del desarrollador, siendo este una réplica exacta del encontrado en los
servidores al que acceden los usuarios. Con esto conseguimos poder generar un
entorno limpio, con todos y cada uno de los elementos involucrados así como un
conjunto de datos mínimo con el que poder empezar a trabajar. Como hito, un desarrollador
debe poder empezar a trabajar en este proyecto en menos de un cuarto de hora.

\subsection{Desarrollo de punto de entrada de información:} Desarrollo de una
\acs{API} \acs{REST} como punto de entrada y salida de información proveniente
de sensores. Debe estar debidamente protegida para que usuarios no autorizados no
puedan acceder a dichos recursos. Se debe maximizar el rendimiento de esta
parte, dado que es crucial a la hora de evitar cuellos de botella y posibles
problemas ante altas tasas de conexiones. 

\subsection{Desarrollo de la aplicación web:} Desarrollo de una aplicación web
donde poder gestionar los datos recibidos por el usuario, tal aplicación deberá
poder hacer búsquedas dentro de estos datos así como permitirnos un mínimo
análisis.

\subsection{Despliegue y configuración de la infraestructura básica:} Desplegar
la infraestructura necesaria para hacerlo accesible a cualquier usuario, desde
el dominio hasta los servidores donde se alojara la aplicación. Se intentara
hacerlo del modo más transparente, dejando la posibilidad de, en un futuro,
poder mejorar la infraestructura en caso de ser necesario. 


\subsection{Creación y gestión de la documentación:} Dado que esta aplicación
será usada por personas externas al proyecto, debemos generar los documentos necesarios para informarles de cómo
hacer uso de los distintos puntos de acceso a la aplicación. Esta documentación
también incluye algún ejemplo con alguna plataforma hardware de
sensorización.
Por otro lado, deberemos disponer de una /acs{API} interna para el
frontend, por lo que debemos tener información para sincronizar los cambios
entre la aplicación y su interfaz web. Por otro lado, también será necesario
documentar casos básicos dentro del lenguaje seleccionado, desde como crear
cualquier tipo de elemento, hasta como generar un test automático para tal, se
debe ilustrar con algún ejemplo interno y explicar cada una de las partes, dado
que el desarrollador no tiene el deber de conocer el lenguaje delecionado.

\subsection{Monitorización de la plataforma:} Tener conocimiento del estado de toda la
plataforma en cada momento, así como recibir alertas en caso de fallo ó bajo
rendimiento. Esto es importante, dado que debemos anticiparnos a posibles picos
de tráfico además de validar la integridad de los mismos. Siendo una aplicación
para terceros, la disponibilidad es clave. Si no obtenemos una disponibilidad
mínima, los usuarios rechazaran nuestra plataforma.

\subsection{Integración continua y despliegue automático:} Siempre que un
desarrollador termine un trabajo, este se comprobará mediante una serie de
tests, si todos estos tests son válidos y obtienen un resultado favorable, se
llevará a cabo un proceso automático de despliegue a los servidores. Con esto
conseguimos facilitar tal proceso, así como aportar un control de calidad previo
a mover estos cambios a cualquier entorno.

\subsection{Pruebas de estrés:} Generar un modelo de pruebas de estrés que nos
permita evaluar el resultado final del rendimiento de la plataforma así como su
estabilidad y la corrección de cada uno de los puntos.

\subsection{Búsqueda de partners:} Búsqueda de socios tecnológicos con los que
probar la aplicación una vez esté en el mercado, permitiendo un posicionamientoe
más rápido así como una prueba inicial de las cualidades del servicio en un
entorno real.

%Los objetivos específicos son las partes independientes del proyecto que tienen
%valor por si mismas.

%Por ejemplo, si el objetivo general fuera destruir una flota enemiga, los
%objetivos específicos podrían ser: hundir el portaaviones, inutilizar las
%torretas de los destructores, eliminar los cazas enemigos, etc.

%Los objetivos específicos no son tareas; análisis, diseño, etc. no tienen valor
%intrínseco para el cliente, si por ejemplo el proyecto se cancela en la fase de
%diseño no se le entrega nada de valor al cliente, luego no se cubre ningún
%objetivo.

%No se deben confundir los objetivos del proyecto con los objetivos del
%alumno. Indicar como objetivo que el alumno va a aprender o a estudiar
%determinada disciplina o herramienta no aporta nada al cliente. Deben ser
%entregables que el cliente puede valorar y por los que estaría dispuesto
%a pagar. Resumiendo, son \textbf{objetivos}, no subjetivos.

% Local Variables:
%  coding: utf-8
%  mode: latex
%  mode: flyspell
%  ispell-local-dictionary: "castellano8"
% End:
