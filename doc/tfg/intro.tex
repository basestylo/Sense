\chapter{Introducción}

\drop{E}{sto} se llama «letra capital» y debería utilizarse únicamente al
comienzo de capítulo como artificio decorativo. Para que resulte estéticamente
adecuada, este primer párrafo debería tener más del doble de líneas de lo que
ocupe verticalmente la letra capital (dos en este caso). El capítulo de
introducción debe dar al lector una perspectiva básica ---pero completa--- del
problema que se pretende abordar, pero también de la estrategia y enfoque que el
autor propone para su resolución. El lector debería poder determinar si este
documento le interesa leyendo únicamente la introducción.


\drop{D}ia a dia miles de sensores toman datos de nuestro entorno, ya sea para controlar las luces de un semaforo, manejar una cadena de montaje, o predecir el
tiempo, todos esos sensores dan información del medio en el que se
encuentran. Toda esta tendencia no hace nada mas que crecer, la cantidad de sensores se ve incrementada de forma
exponencial, tanto que recientemente podemos haber escuchado hablar de conceptos
como el Internet de las Cosas (en ingles Internet of Things o Iot), M2M o de las
smart cities, todos estos conceptos tienen en común la necesidad de la toma de
datos por parte de sensores y la posterior comunicación para notificar del valor
obtenido, todo esto hace que millones de lecturas deban ser manejadas de una
forma eficiente y sobre todo a un bajo coste. Recientemente hemos podido ver
como el número de plataformas ligadas al almacenamiento de datos relacionados con IoT ha crecido de forma tambien exponencial.

La mayoria de estos solo te permiten, dentro del plan gratuito, unas cuotas de
datos irrisorias para cualquier pequeño conjunto de sensores, teniendo que
sacrificar resolución, ademas de tener una baja persistencia, ya que los datos son borrados pasado un tiempo. Una vez salimos del plan gratuito, los precios pueden llegar a dispararse.

El conjunto de todos estos detalles es lo que da forma a este proyecto, donde se
busca obtener una solución eficiente, con el minimo coste y con una maxima tasa
de disponibilidad. Aunar todos estos requisitos es
complicado, por lo que, se requerira de un exaustivo analisis de las diferentes
opciones existentes eligiendo la mejor opción posible de todas las existentes
dentro de cada categoria, este proyecto llevara a cada tecnologia involucrada a
su maximo exponente.

Como punto final, el proyecto sera planificado mediante metodologias agiles,
mediante la planificación de pequeños sprints, desarrollo guiado por
tests, integración continua y despliege continuo 

Nota de prensa IoT, IBM: http://www-03.ibm.com/press/es/es/presskit/50110.wss
plataformas: AWS IoT, Azure IoT, Google Cloud IoT, IBM.
Servicios: Thingspeak, Carriots, Adafruit IO, Sentilo, Devicehive, Smart Cities as a Service, Pubnub, Thingworx, Temboo, Blink, Thethings, Ubidots, 2lemetry, Onion Cloud.

\section{Título del proyecto}

En la portada ---y otras páginas de presentación--- el nombre o título del
proyecto debe aparecer sin comillas, cursiva u otros formatos. Si se cita el
título de otra obra, o el nombre de un capítulo sí debe aparecer entre
comillas. Por cierto, las comillas que deben usarse en castellano son las
«latinas», dejando las ``inglesas'' para los raros casos en los que aparezca una
cita en el cuerpo otra~\cite{sousa}.


\section{Estructura del documento}

Pueden incluirse aquí una sección con algunos consejos para la lectura del
documento dependiendo de la motivación o conocimientos del lector.  También
puede ser útil incluir una lista con el nombre y finalidad de cada uno de los
capítulos restantes.

\begin{definitionlist}
\item[Capítulo \ref{chap:antecedentes}: \nameref{chap:antecedentes}] Explica herramientas
  y aspectos básicos de edición con \LaTeX.
\item[Capítulo \ref{chap:objetivos}: \nameref{chap:objetivos}] Finalidad y justificación
  (con todo detalle) del presente documento.
\end{definitionlist}


\section{Más texto para que ocupe varias páginas}

\blindtext
\blinditemize[4]
\blindmathpaper

\section{Otra sección}

\blindtext


% Local Variables:
%  coding: utf-8
%  mode: latex
%  mode: flyspell
%  ispell-local-dictionary: "castellano8"
% End:
