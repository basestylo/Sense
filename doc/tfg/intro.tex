\chapter{Introducción}

%\drop{E}{sto} se llama «letra capital» y debería utilizarse únicamente al
%comienzo de capítulo como artificio decorativo. Para que resulte estéticamente
%adecuada, este primer párrafo debería tener más del doble de líneas de lo que
%ocupe verticalmente la letra capital (dos en este caso). El capítulo de
%introducción debe dar al lector una perspectiva básica ---pero completa--- del
%problema que se pretende abordar, pero también de la estrategia y enfoque que el
%%autor propone para su resolución. El lector debería poder determinar si este
%documento le interesa leyendo únicamente la introducción.


\drop{E}{l} ser humano siempre ha tenido la curiosidad de explorar su entorno,
conocerlo más de cerca y analizar los sucesos que ocurren en este, se podria
decir que se ve atraído por lo que no conoce, e intenta clasificarlo y sacar
conclusiones, en el pasado, todo este proceso de obtención de información se
realizaba de manera manual, pero por suerte, a dia de hoy tenemos elementos que
se encargan se esta labor, son los sensores, que nos permiten obtener datos de
una forma programática para procesarlos posteriormente.

En los últimos años podemos presenciar un creciente interés en la
tematica relacionada con la sensorización, existe una gran cantidad de
comunidad en torno a esta y esta surgiendo infinidad de plataformas hardware que
ven en la sensorización su mayor fuerza, como hemos dicho anteriormente, estos
dispositivos tienen la capacidad de ofrecernos información del entorno, pudiendo
esta servir en infinidad de casos, como por ejemplo: 
\begin{itemize}
\item Análisis del clima
\item Detección temprana de catástrofes
\item Gestión de una cadena de montaje.
\item Monitorización de los niveles de radiación en una central nuclear
\end{itemize}

Todos estos sensores son cada vez mas comunes en nuestro entorno, tanto que, ciudades enteras se
nutren de estos para la toma de decisiones críticas, un ejemplo de esto es la
capital de España, Madrid, cuyos sensores de polución han causado más de una
noticia en televisión, con los datos recogidos por estos sensores se deciden
temas tan importantes que afectan a la salud de estos ciudadanos.

Toda esta tendencia no hace nada más que crecer, la cantidad de
sensores se ve incrementada de forma exponencial, tanto que recientemente
podemos haber escuchado hablar de conceptos como el Internet de las Cosas (en
inglés Internet of Things o \acs{IoT}), \acs{M2M} (Machine to machine) o de las Smart
Cities. Todos estos
conceptos tienen en común la necesidad de la toma de datos por parte de sensores
y la posterior comunicación para notificar del valor obtenido, todo esto hace
que millones de lecturas tengan que ser manejadas de una forma eficiente y sobre todo
a un bajo coste.

\newpage
Por otra parte, toda esta tendencia ha venido acompañada de una serie de
plataformas, estas tratan de solucionar los problemas de comunicación así como
de almacenamiento hablados anteriormente, en concreto hemos podido ver como el número de plataformas ligadas al almacenamiento de datos relacionados con \acs{IoT} ha crecido de forma también exponencial.
La mayoría de estos solo te permiten, dentro del plan gratuito, unas cuotas de
datos irrisorias para cualquier pequeño conjunto de sensores, teniendo que
sacrificar resolución, ademas de tener una baja persistencia, ya que los datos son borrados pasado un tiempo. Una vez salimos del plan gratuito, los precios pueden llegar a dispararse.

El conjunto de todos estos detalles es lo que da forma a este proyecto, donde se
busca obtener una solución eficiente, con el mínimo coste y con una máxima tasa
de disponibilidad. Aunar todos estos requisitos es
complicado, por lo que, se requerirá de un exhaustivo análisis de las diferentes
opciones existentes eligiendo la mejor opción posible de todas las existentes
dentro de cada categoría. Este proyecto llevará a cada tecnología involucrada a
su máximo exponente, conceptos como el uso hibrido de bases de datos, que nos
permiten seleccionar para cada momento donde almacenar la información en función
del mejor tiempo de respuesta,
tecnologias web bien probadas durante años en el campo de las telecomunicaciones
formarán parte de nuestras herramientas para poner solución al problema de tener millones de
sensores conectados.

Otro punto a destacar es el que todo lo desarrollado será software libre, donde
cualquier persona podrá consultar como funciona, mejorarlo, o incluso, realizar
su propia solución. Con todo esto se busca crear una nueva plataforma
open-source, uno de los pocos casos en este campo, buscando poner un punto de
partida donde, en un futuro, exista un proyecto colaborativo que se maneje por
la propia comunidad.

Como punto final, el proyecto será planificado mediante metodologías ágiles,
mediante la planificación de pequeños sprints, desarrollo guiado por
test y integración continua, así como despliegue continuo, pudiendo controlar
cada una de las fases del software de una forma exhaustiva.

Con todo esto, este proyecto pretende ser una alternativa respecto a las actuales plataformas de
gestión de datos relacionados con la sensorización. <<Sense>>  nace como un nuevo
candidato en un mundo en el que, sus actuales competidores, no son lo suficiente
eficientes para poder mantener los costes a un nivel bajo. Partimos con la
ventaja de aprender de sus errores, y con la posibilidad de mejorar lo
actualmente existente.

%Nota de prensa IoT, IBM: http://www-03.ibm.com/press/es/es/presskit/50110.wss
%plataformas: AWS IoT, Azure IoT, Google Cloud IoT, IBM.
%Servicios: Thingspeak, Carriots, Adafruit IO, Sentilo, Devicehive, Smart Cities as a Service, Pubnub, Thingworx, Temboo, Blink, Thethings, Ubidots, 2lemetry, Onion Cloud.

%\section{Título del proyecto}

%En la portada ---y otras páginas de presentación--- el nombre o título del
%proyecto debe aparecer sin comillas, cursiva u otros formatos. Si se cita el
%título de otra obra, o el nombre de un capítulo sí debe aparecer entre
%comillas. Por cierto, las comillas que deben usarse en castellano son las
%«latinas», dejando las ``inglesas'' para los raros casos en los que aparezca una
%cita en el cuerpo otra~\cite{sousa}.


%\section{Estructura del documento}

%Pueden incluirse aquí una sección con algunos consejos para la lectura del
%documento dependiendo de la motivación o conocimientos del lector.  También
%puede ser útil incluir una lista con el nombre y finalidad de cada uno de los
%capítulos restantes.

%\begin{definitionlist}
%\item[Capítulo \ref{chap:antecedentes}: \nameref{chap:antecedentes}] Explica herramientas
%  y aspectos básicos de edición con \LaTeX.
%\item[Capítulo \ref{chap:objetivos}: \nameref{chap:objetivos}] Finalidad y justificación
%%  (con todo detalle) del presente documento.
%\end{definitionlist}


%\section{Más texto para que ocupe varias páginas}

%\blindtext
%\blinditemize[4]
%\blindmathpaper

%\section{Otra sección}

%\blindtext


% Local Variables:
%  coding: utf-8
%  mode: latex
%  mode: flyspell
%  ispell-local-dictionary: "castellano8"
% End:
